% additional options: [seceqn,secthm,crcready,onecolumn]
\documentclass[sat]{iosart2x}

%% Packages
\usepackage{dcolumn}
%\usepackage{endnotes}

%% Definitions
\newcolumntype{d}[1]{D{.}{.}{#1}}

%% Article Info
\firstpage{1}
\lastpage{5}
\volume{1}
\pubyear{2016}

\begin{document}
\begin{frontmatter} % The preamble begins here.

%
%\pretitle{Pretitle}
\title{Instructions for the preparation of a camera-ready paper in \LaTeX\thanks{Footnote in title.}}

\runtitle{Instructions for the preparation of a camera-ready paper in \LaTeX}
%\subtitle{Subtitle}


\author[A]{\inits{F.}\fnms{First} \snm{Author}\ead[label=e1]{editorial@iospress.nl}}
\author[B]{\inits{S.}\fnms{Second} \snm{Author}\ead[label=e2]{second@somewhere.com}}
\author[B]{\inits{T.}\fnms{Third} \snm{Author}\ead[label=e3]{third@somewhere.com}}
\address[A]{Journal Production Department, \orgname{IOS Press}, Nieuwe Hemweg 6b, 1013 BG, Amsterdam, \cny{The~Netherlands}}
\address[B]{Department first, \orgname{University or Company name},
Abbreviate US states, \cny{Country}}

\begin{abstract}
The abstract should be clear, descriptive, self-explanatory and no longer than 200 words. It should also
be suitable for publication in abstracting services. Do not include references or formulae in the abstract.
\end{abstract}

\begin{keyword}
\kwd{Keyword one}
\kwd{keyword two}
\kwd{keyword three}
\kwd{keyword four}
\kwd{keyword five}
\end{keyword}
\end{frontmatter}




\section{Introduction}
This document provides instructions for style and layout of a double column journal article.
The instructions are designed for the preparation of a camera-ready and accepted paper in \LaTeX{} and should be read carefully.
Prepare your paper in the same style as used in this sample pdf file.
Unfortunately IOS Press can not give technical support on the template.
These instructions also contain the necessary information for manual editing. Authors should realize that once the accepted
paper is submitted for publication in the journal, the print and online version will be almost identical to the paper submitted
for publication, except for the pagination, volume, and copyright year
information.

Manuscripts must be written in English. Authors whose native language is not English are recommended to seek the advice of a native English speaker,
if possible, before submitting their manuscripts. The pages in the manuscript should not be numbered and in the text no reference should be made
to page numbers; if necessary, one may refer to sections. Try to avoid excessive use of italics and bold face.


\section{Structure of the paper}
\subsection{Title page}

The title page should provide the following information:

\begin{itemize}
\item Title (should be clear, descriptive and not too long).
\item  Name(s) of author(s); in the case of multiple authors always indicate who is the corresponding
author.
\item  Full affiliation(s). Names of main institution and country should be given within respective tags: \verb|\orgname{}|, \verb|\cny{}|.
\item  Complete address of corresponding author, including tel. number and e-mail address.
\item  Abstract should be clear, descriptive, self-expla\-na\-tory and no longer than 200
words. It should also be suitable for publication in abstracting services.
\item  Up to five keywords.
\end{itemize}

\subsection{Body of the text}
\begin{itemize}
\item The use of first persons (i.e., ``I'', ``we'', ``their'', possessives, etc.) should be avoided,
and can preferably be expressed by the passive voice or other ways. This also applies to the Abstract.
\item A research paper should be structured in terms of four parts, each of which may comprise of multiple sections:
\begin{itemize}
\item Part One is problem description/definition, and a literature review upon the state of the
art
\item Part Two is methodological formulation and/or theoretical development (fundamentals, principle and/or approach,
etc.)
\item Part Three is prototyping, case study or experiment;
\item Part Four is critical evaluation against related works, and the
conclusion.
\end{itemize}
\item A survey paper may skip Part Three, but should multiply Part Two and elaborate Parts One and Four.
\item An application paper may lightly touch Part Two but should elaborate Part Three, with Parts One and Four similar to the a
research paper.
\end{itemize}

In any article it is unnecessary to have an arrangement statement at the beginning (or end) of every \mbox{(sub-)} section.
Rather, a single overall arrangement statement about the whole paper can be made at the end of the Introduction section.


\section{Typographical style and layout}

\subsection{Type area}

The \texttt{iosart2x.cls} document class has been designed to produce
the right layout from your \LaTeX{} input. Authors are requested to
strictly follow these instructions. \emph{The provided class file
iosart2x must not be changed}.

The text output area is automatically set within an area of particular journal layout measurements.
Please do not use any
\LaTeX{} or \TeX{} commands that affect the layout or formatting of
your document (i.e. commands like \verb|\textheight|,
\verb|\textwidth|, etc.).

\subsection{Font}

The font type for running text (body text) is 10/11~pt Times New Roman
depending on the style of particular journal.
There is no need to code normal type (roman text). For literal text, please use
\texttt{type\-writer} (\verb|\texttt{}|)
or \textsf{sans serif} (\verb|\textsf{}|). \emph{Italic} (\verb|\emph{}|)
or \textbf{boldface} (\verb|\textbf{}|) should be used for emphasis.


\subsection{General layout}

Use single (1.0) line spacing throughout the document. For the main
body of the paper use the commands of the standard \LaTeX{}
``article'' class. You can add packages or declare new \LaTeX{}
functions if and only if there is no conflict between your packages
and the \texttt{iosart2x.cls}.

Always give a \verb|\label| where possible and use \verb|\ref| for cross-referencing.

Class option ``\texttt{onecolumn}'' should be used only if it is
necessary for the layout of your article:\hfill\break
\verb|\documentclass[onecolumn]{iosart2x}|.

Use class option ``\texttt{crcready}'' in order to remove the page numbers from your article.


\subsection{Title page}

Use sentence case for the title.

Use \verb|\thanks{}| command for footnotes in\break \verb|\title|,
\verb|\author| and \verb|\address| commands.

Do not use capitals for author's surname. Add ``and'' before the last
author. Do not add a period after the last keyword.


\subsection{(Sub-)section headings}

Use the standard \LaTeX{} commands for headings: \verb|\section|, \verb|\subsection|, \verb|\subsubsection|, \verb|\paragraph|.
Headings will be automatically numbered.

\subsection{Footnotes and endnotes}
Footnotes should only be used if absolutely essential.
In most cases it is possible to incorporate the information in the text.
If used, they should be kept as short as possible.
The footnotes are numbered automatically. If the footnotes take up more space than roughly 10\% of the type area,
list them as endnotes, before the References. Footnotes within the text should be coded with the command
\verb|\footnote{Text}|,
 endnotes with the command \verb|\endnote{Text}|.
The command \verb|\theendnotes| prints out the endnotes.
The additional package \texttt{endnotes} is required for the endnotes (\verb|\usepackage{endnotes}|).
If you do not have it in your \LaTeX{} environment,
see your system manager, or download them from \url{http://www.ctan.org}.



\subsection{References}

References should be collected at the end of your paper (environment
\verb|thebibliography|). References should be listed alphabetically in the style presented in the section \textbf{References} at the
end of these instructions. Use the command \verb|\cite| to refer to the entries in the bibliography so that your
accumulated list corresponds to the citations made in the text body.

\begin{figure}[t]
\leavevmode\fbox{\parbox[b][30mm][s]{30mm}{
\vfill\footnotesize Please include graphics in Encapsulated
PostScript (.eps) format\vfill}}
\caption{Figure caption.}
\end{figure}


\subsection{Figures}
\subsubsection{General remarks on figures}
The text should include references to all figures.
Refer to figures in the text as Figure 1, Figure 2, etc., \textbf{not} with the section number included,
e.g. Figure 2.3, etc. Do not use the words ``below'' or ``above'' when referring to the
figures.

Do not collect figures at the back of your article, but incorporate them in the text.

Position figures at the top or bottom of a page, near the paragraph in which the figure is first mentioned.
Figures should not have text wrapped alongside.

Each figure should have a self-explanatory caption. Place the figure caption \textit{below} the figure.

All figures coded with \verb|figure| and \verb|\caption| will be numbered automatically.

On maps and other figures where a scale is needed, use bar scales rather than numerical ones of the type 1:10,000.

%They will be numbered automatically.

\subsubsection{Quality of illustrations}\label{s3.8.2}
Use only Type I fonts for the lettering in illustrations. Do \textit{not} use illustrations taken from the Internet.
The resolution of images intended for viewing on a screen is not sufficient for the printed version of the journal.
If you are incorporating screen captures, keep in mind that the text may not be legible after reproduction
(using a screen capture tool, instead of the Print Screen option of PC's, might help to improve the
quality). Preferably the images should be enclosed as PostScript
files -- best as EPS data.


\begin{table*}[htb]
\caption{Table caption} \label{t1}
\begin{tabular}{ll d{1.3} d{1.3} d{1.3} d{1.3} d{1.3} d{1.3}}
\hline
Dataset&Models&\multicolumn{1}{l}{$\alpha_1$}&
\multicolumn{1}{l}{$\alpha_2$}&
\multicolumn{1}{l}{$\alpha_3$}&
\multicolumn{1}{l}{$\alpha_4$}&
\multicolumn{1}{l}{$\alpha_5$}&
\multicolumn{1}{l}{$\alpha_6$}\\
\hline
CSDS&Linear     &0.164&0.22&0.123&0.3&0.200&0.258\\
& Logistic      &0.189&0.155&0.157&0.201&0.154&0.144\\[6pt]
KCDS&Linear     &0.155&0.183&0.160&0.218&0.176&0.156\\
& Logistic      &0.187&0.125&0.151&0.184&0.187&0.125\\
\hline
\end{tabular}
\end{table*}



\begin{itemize}
\item Line art should have a minimum resolution of 600 dpi;  %, save as EPS or TIFF
\item  grayscales (incl photos) should have a minimum resolution of 300 dpi (no lettering), or 500 dpi (when there is lettering); %save as tiff
\item  do not save figures as JPEG, this format may lose information in the
process;
\item  do not use figures taken from the Internet, the resolution will be too low for
printing;
\item  do not use color in your figures if they are to be printed in black \& white, as this will reduce the print
quality (note that in software the default is often color, so you should change the
settings);
\item  for figures that should be printed in color, please send a CMYK encoded EPS.% or TIFF
\end{itemize}

\subsubsection{Color figures}
It is possible to have figures printed in color,
provided the cost of their reproduction is paid for by the author.
Please contact \email{editorial@iospress.nl} for a quotation if you wish to have figures printed in color.
See Section~\ref{s3.8.2} for the required file formats.

\subsection{Tables}

The text should include references to all tables.
Refer to tables in the text as Table 1, Table 2, etc., not with the section number included,
 e.g. Table 2.3, etc. Do not use the words ``below'' or ``above'' referring to the
 tables.

Position tables at the top or bottom of a page, near the paragraph in which the table is first mentioned.
Tables should not have text wrapped alongside.


Code your tables using \LaTeX{} environments \verb|table| and
\verb|tabular|. Each table should have a brief and self-explanatory caption that should be put above the
table. Do not use the period at the end of the table caption. If the table does not fit into one column it may be placed across both columns
using \verb-\begin{table*}-
so that it
appears at the top of a page.




All tables coded with \verb|table| and \verb|\caption| will be numbered automatically.

Column headings should be brief, but sufficiently explanatory.
Standard abbreviations of units of measurement should be added between parentheses.
Vertical lines should not be used to separate columns. Leave some extra space between the columns instead.
Any explanations essential to the understanding of the table should be given in footnotes at the bottom of the table.
SI units should be used, i.e., the units based on the metre, kilogramme, second, etc.

Tables should be presented in the form shown in
Table~\ref{t1}.  Their layout should be consistent
throughout.





\subsection{Equations}

Do not put in equation numbers, since this is taken care of
automatically. The equation numbers are always consecutive and are
printed in parentheses flush with the right-hand margin of the text and
level with the last line of the equation. For multi-line equations, use
the {\tt eqnarray} environment. Refer to equations in the text as Eq. (1), Eqs. (3) and (5).


\section{Fine tuning}
\subsection{Type area}
Check once more that all the text and illustrations are inside the type area and
that the type area is used to the maximum.

\subsection{Capitalization}
Use sentence case in the title and the headings.

\subsection{Page numbers and running headlines}
You do not need to include page numbers or running headlines. These elements will be added by the publisher.

\section{Submitting the paper}
Submit the following to the journal Editorial office, online submission form or Editor-in-Chief (whichever is applicable for the journal):

\begin{enumerate}
\item The main \LaTeX{} document as well as other required files (eps, etc.).
\item Identical high resolution PDF file with all fonts embedded. First produce a Postscript file from \LaTeX{}
with DVIPS version 5.56 or higher. The option "-M" (don't make fonts) should be indicated.
Use Adobe Acrobat Distiller and choose the job option Press-Optimized.) Please make sure you do not submit more than one version of any item.
\end{enumerate}

\section{Copyright}
Authors submitting a manuscript do so in the understanding that they have read and agreed to the terms
of the IOS Press Author Copyright Agreement available through \url{http://www.iospress.nl}.

\subsection{Quoting from other publications}
An author, when quoting from someone else's work or when considering reproducing figures or tables from a book or journal article,
 should make sure that he/she is not infringing a copyright. Although in general an author may quote from other published works,
 he should obtain permission from the holder of the copyright if he wishes to make substantial extracts or to reproduce tables,
  plates or other figures. If the copyright holder is not the author of the quoted or reproduced material, it is recommended
  that the permission of the author should also be sought. Material in unpublished letters and manuscripts is also protected
  and must not be published unless permission has been obtained. Submission of a paper will be interpreted as a statement that
the author has obtained all the necessary permission. A suitable acknowledgement of any borrowed material must always be made.

\section{How to order reprints, a pdf file, journals, or IOS Press
books}
The corresponding author of a contribution to the journal is entitled to receive a complimentary PDF copy of the published article, unless otherwise stated.
Free copies will not be provided for conference proceedings and abstract issues. An order form for reprints,
additional journal copies or a pdf file will be provided along with the pdf proof.
If you wish to order reprints of an earlier published article, please contact the publisher for a quotation at IOS Press, E-mail: \email{editorial@iospress.nl}.
An author is entitled to 25\% discount on IOS Press books.

%\begin{ack}
%\end{ack}

\begin{acks}
Please include your acknowledgements in \verb|acks| environment. For short acknowledgment you may use \verb|ack| environment.
\end{acks}

\begin{appendix}
\section{Appendix title}
Appendices should be provided in \verb|appendix| environment, before References.

\subsection{First subsection}
Use the standard \LaTeX{} commands for headings in Appendix.

\subsection{Second subsection}
Headings will be automatically numbered. 
\end{appendix}


\begin{thebibliography}{0}

\bibitem{1} L. Lamport, \textit{\LaTeX{} User's Guide \& Reference Manual},
 Addison Wesley Publishing Co, 1985.

\bibitem{2} B. Newman and E.T. Liu, Perspective on BRCA1, \textit{Breast Disease} \textbf{10} (1998), 3--10. \bid{doi={10.3233/BD-1998-101-203}}

\bibitem{3} D.F. Pilkey, Happy conservation laws, in: \textit{Neural Stresses}, J. Frost, ed., Controlled Press, Georgia, 1995, pp. 332--391.

\bibitem{4} E. Wilson, Active vibration analysis of thin-walled beams, Ph.D. Dissertation, University of Virginia, 1991.

\end{thebibliography}


\end{document}
